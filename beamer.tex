\documentclass[aspectratio=169]{beamer}


\usepackage[english]{babel}
\usepackage{amsthm}
\usepackage{amssymb}
\usepackage{graphicx}
\usetheme{Bergen}
\usecolortheme{beaver}

\title{Create slides using beamer}
\author{Jaydeep Paul}
\date{\today}
\begin{document}
\maketitle
\begin{frame}
\tableofcontents
\end{frame}
\section{Introduction}
\begin{frame}[t]{Title}
Theorems can easily be defined:

\begin{theorem}
Let \(f\) be a function whose derivative exists in every point, then \(f\) is a continuous function.
\end{theorem}

\end{frame}

\begin{frame}
And a consequence of theorem \ref{pythagorean} is the statement in the next 
corollary.

\begin{corollary}
There's no right rectangle whose sides measure 3cm, 4cm, and 6cm.
\end{corollary}

You can reference theorems such as \ref{pythagorean} when a label is assigned.

\begin{lemma}
Given two line segments whose lengths are \(a\) and \(b\) respectively there is a 
real number \(r\) such that \(b=ra\).
\end{lemma}

\begin{theorem}
Let \(f\) be a function whose derivative exists in every point, then \(f\) is a continuous function.
\end{theorem}
\end{frame}

\begin{frame}

\end{frame}

\end{document}
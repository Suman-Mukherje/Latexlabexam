\documentclass{report}
\usepackage[margin=2cm]{geometry}
\usepackage[english]{babel}
\usepackage{amsthm}
\usepackage{amssymb}
\usepackage{graphicx}

\newtheorem{theorem}{MyTheorem}[chapter]
\newtheorem{corollary}{Corollary}[theorem]
\newtheorem{lemma}[theorem]{Lemma}

\theoremstyle{definition}
\newtheorem{definition}{Definition}[chapter]

\theoremstyle{remark}
\newtheorem*{remark}{Remark}

\usepackage{caption}
\usepackage{subcaption}

\begin{document}

\chapter{Theorem and definition}
\section{Introduction}
Theorems can easily be defined:

\begin{theorem}
Let \(f\) be a function whose derivative exists in every point, then \(f\) is a continuous function.
\end{theorem}

\section{Second}
\begin{theorem}[Pythagorean theorem]
\label{pythagorean}
This is a theorem about right triangles and can be summarised in the next 
equation \[ x^2 + y^2 = z^2 \]
\end{theorem}

And a consequence of theorem \ref{pythagorean} is the statement in the next 
corollary.

\begin{corollary}
There's no right rectangle whose sides measure 3cm, 4cm, and 6cm.
\end{corollary}

You can reference theorems such as \ref{pythagorean} when a label is assigned.

\begin{lemma}
Given two line segments whose lengths are \(a\) and \(b\) respectively there is a 
real number \(r\) such that \(b=ra\).
\end{lemma}

\begin{theorem}
Let \(f\) be a function whose derivative exists in every point, then \(f\) is a continuous function.
\end{theorem}

\begin{remark}
This statement is true, I guess.
\end{remark}

And the next is a somewhat informal definition

\begin{definition}[Sample]
This is a sample definition.
\end{definition}

\renewcommand\qedsymbol{$\blacksquare$}

\begin{proof}
This is proved.
\end{proof}

\renewcommand\qedsymbol{QED}

\begin{proof}
This is proved.
\end{proof}

\chapter{Images and Tables}

\begin{figure}[!h]
     \centering
     \begin{subfigure}[b]{0.3\textwidth}
         \centering
         \includegraphics[width=\textwidth]{img}
         \caption{One}
         \label{fig:onesub}
     \end{subfigure}
     \hfill
     \begin{subfigure}[b]{0.3\textwidth}
         \centering
         \includegraphics[width=\textwidth]{img}
         \caption{Two}
         \label{fig:twosub}
     \end{subfigure}
     \hfill
     \begin{subfigure}[b]{0.3\textwidth}
         \centering
         \includegraphics[width=\textwidth]{img}
         \caption{Three}
         \label{fig:threesub}
     \end{subfigure}
        \caption{Three simple images}
        \label{fig:three imgs}
\end{figure}

\begin{table}[h]
    \begin{subtable}[h]{0.45\textwidth}
        \centering
        \begin{tabular}{l | l | l}
        Day & Max Temp & Min Temp \\
        \hline \hline
        Mon & 20 & 13\\
        Tue & 22 & 14\\
        Wed & 23 & 12\\
        Thurs & 25 & 13\\
        Fri & 18 & 7\\
        Sat & 15 & 13\\
        Sun & 20 & 13
       \end{tabular}
       \caption{First Week}
       \label{tab:week1}
    \end{subtable}
    \hfill
    \begin{subtable}[h]{0.45\textwidth}
        \centering
        \begin{tabular}{l | l | l}
        Day & Max Temp & Min Temp \\
        \hline \hline
        Mon & 17 & 11\\
        Tue & 16 & 10\\
        Wed & 14 & 8\\
        Thurs & 12 & 5\\
        Fri & 15 & 7\\
        Sat & 16 & 12\\
        Sun & 15 & 9
        \end{tabular}
        \caption{Second Week}
        \label{tab:week2}
     \end{subtable}
     \caption{Max and min temps recorded in the first two weeks of July}
     \label{tab:temps}
\end{table}

\end{document}
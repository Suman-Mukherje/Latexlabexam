\documentclass[a4paper,12pt]{report}

\usepackage[margin=2cm]{geometry}
\usepackage{hyperref}
\usepackage{amsmath,amsfonts}

\usepackage{graphicx}
\graphicspath{{images/}}

\title{First Lecture of \LaTeX}
\author{Jaydeep Paul}
\date{\today}
\begin{document}
	\maketitle
	\tableofcontents
	\chapter{Introduction}
	
	\hspace{2em}LaTeX [2][Note 1] often stylized as LATEX) is a software system for document preparation.[3] When writing, the writer uses plain text as opposed to the formatted text found in WYSIWYG word processors like Microsoft Word, LibreOffice Writer and Apple Pages. The writer uses markup tagging conventions to define the general structure of a document, to stylise text throughout a document (such as bold and italics), and to add citations and cross-references. A TeX distribution such as TeX Live or MiKTeX is used to produce an output file (such as PDF or DVI) suitable for printing or digital distribution.\\
	
	LaTeX is widely used in academia[4][5] for the communication and publication of scientific documents in many fields, including mathematics, computer science, engineering, physics, chemistry, economics, linguistics, quantitative psychology, philosophy, and political science. It also has a prominent role in the preparation and publication of books and articles that contain complex multilingual materials, such as Arabic and Greek.[6] LaTeX uses the TeX typesetting program for formatting its output, and is itself written in the TeX macro language.\\
	
	LaTeX can be used as a standalone document preparation system, or as an intermediate format. In the latter role, for example, it is sometimes used as part of a pipeline for translating DocBook and other XML-based formats to PDF. The typesetting system offers programmable desktop publishing features and extensive facilities for automating most aspects of typesetting and desktop publishing, including numbering and cross-referencing of tables and figures, chapter and section headings, graphics, page layout, indexing and bibliographies. \\
	
	Like TeX, LaTeX started as a writing tool for mathematicians and computer scientists, but even from early in its development, it has also been taken up by scholars who needed to write documents that include complex math expressions or non-Latin scripts,[7] such as Arabic, Devanagari and Chinese.[8]
LaTeX is intended to provide a high-level, descriptive markup language that accesses the power of TeX in an easier way for writers. In essence, TeX handles the layout side, while LaTeX handles the content side for document processing. LaTeX comprises a collection of TeX macros and a program to process LaTeX documents, and because the plain TeX formatting commands are elementary, it provides authors with ready-made commands for formatting and layout requirements such as chapter headings, footnotes, cross-references and bibliographies.
LaTeX was originally written in the early 1980s by Leslie Lamport at SRI International.[9] The current version is LaTeX2e (stylised as LATEX2), released in 1994, but updated in 2020. LaTeX3 (LATEX3) has been under long-term development since the early 1990s. LaTeX is free software and is distributed under the LaTeX Project Public License (LPPL).[10]


	\chapter{Image}
	I am inserting an image.\\
	
	\begin{figure}[!h]
		\begin{center}
			\includegraphics[scale=0.3]{test}
			\caption{Our test image of \LaTeX \ logo}
			\label{fig:logo}
		\end{center}
	\end{figure}
	\noindent I have done it. See \ref{fig:logo} above.
	\chapter{Chapter and section}
		\section{Sample}
			\subsection{sub sample}
				\subsubsection{kwfh}
	\chapter{Lists}
		\section{Unodered lists}
		\begin{itemize}
			\item Test item
			\item Another test item
		\end{itemize}
		\section{Ordered lists}
		\begin{enumerate}
			\item Test item
			\item Another test item
			\begin{enumerate}
				
			\item Test item
			\item Another test item
				\begin{enumerate}
				
			\item Test item
			\item Another test item
				\end{enumerate}
				\begin{itemize}
				
			\item Test item
			\item Another test item
				\end{itemize}
			\end{enumerate}
		\end{enumerate}
	\chapter{Math mode}
	\section{Inline math mode}
	I have done it. See \ref{fig:logo} above.
	Our first class of algebra tought us $x+x=2x$.
	
	\noindent $(a+b)^2=a^2+2ab+b^2$ \\
	
	$y=\int^{\infty}_{-\infty} f(x)dx$ \\

	The sum of natural numbers is $\left[\sum^n_{i=1} i = \frac{n*(n+1)}{2}\right]$
	\section{Dedicated Math mode}
	The sum of natural numbers is $$\sum^n_{i=1} i = \frac{n*(n+1)}{2}$$
	\section{Matrix}
	$\begin{matrix}
		1 & 2 \\
		3 & 4
	\end{matrix}$
	
	\chapter{Table}
	\section{Manual Table Creation}
	\begin{table}[!h]
		\begin{center}
		\caption{Our First Sample Table}
		\vspace{1em}
		\begin{tabular}{||c|r|l||} 
			1 & 2 & 3 \\ \hline
			4 & 5 & 6 \\ 
			7 & 8 & 9
		\end{tabular}
		\end{center}
	\end{table}
	
	\section{Online Table Creation}
	\begin{table}[!h]
		\begin{center}
		\caption{Our First Online Sample Table}
		\vspace{1em}
		\begin{tabular}{lll}
1                      & 2                      & 3 \\ \cline{2-2}
\multicolumn{1}{l|}{4} & \multicolumn{1}{l|}{5} & 6 \\ \cline{2-2}
7                      & 8                      & 9
\end{tabular}
		\end{center}
	\end{table}
	
	\chapter{HOw to use bibliography}
	This is a citation \cite{baidoo2023education}\cite{abc}
	\bibliography{ref}
	\bibliographystyle{plain}
\end{document}









